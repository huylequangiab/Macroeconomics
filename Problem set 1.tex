\documentclass[a4paper, 11pt]{article}

%% Language and font encodings
\usepackage[english]{babel}
\usepackage[utf8x]{inputenc}
\usepackage[T1]{fontenc}
\usepackage{libertine} 

%% Sets page size and margins
\usepackage[a4paper,top=2cm,bottom=2cm,left=3cm,right=3cm,marginparwidth=1.75cm]{geometry}

%% Useful packages
\usepackage{amsmath}
\usepackage{graphicx}
\usepackage[colorinlistoftodos]{todonotes}
\usepackage[colorlinks=true, allcolors=blue]{hyperref}
\usepackage{booktabs} 
\usepackage{multirow}

\title{Problem Set 1}
\author{Huy Le Quang}

\begin{document}
\maketitle

\textbf{Question 1:} What kind of questions does Macroeconomics try to answer? 

\textbf{Answer:} 

Macroeconomics studies the forces that influence the economy as a whole, i.e. it tries to explain the performance and development of aggregatedeconomic key variables.

Macroeconomics attempts not only to explain key variables but it tries also to advice policy makershow to cure economic problemsand how to carry out economic policies. \\

\textbf{Question 2:} Describe five macroeconomic key variables. 

\textbf{Answer:} 

Gross Domestic Products (GDP) and its components: consumption (C), investment (I), government purchases (G) and net export (NX) 

Money and monetary aggregates 

Overall price level and inflation 

Unemployment 

Balance of Payment (BoP): aggregates all economic transactions between a country and the rest of the world. \\

\textbf{Question 3:} What are defining characteristics of positive and normative analyses in science? How would you interpret the terms positive and normative analyses in the field of macroeconomics? 

\textbf{Answer:}

Positive analysis: Descriptive, factual statements about the world. Positive analysis uses scientific principles to arrive at objective, testable conclusions. 

Normative analysis: Prescriptive, value-based statements. What policies or institutional arrangements lead to the best outcomes. \\

\textbf{Question 4:} Explain the differences between growth theory and real business cycle theory. 

\textbf{Answer:}

Growth theory: (very) long-run perspective: explains GDP trend. Main driver: Technological innovations 

Real business cycle theory: short run perspective. Economic policy in the short run mainly focuses on price stabilityand low unemployment. \\

\textbf{Question 5:} What is the difference between the trend and the business cycle component of an economic time series? 

\textbf{Answer:}

Trend defines long--run growth of a variable, i.e. GDP (natural or potential GDP = $Y_{n}$)

Business cycle component illustrates a short--run fluctuation of a variable. \\

\textbf{Question 6:}: Explain the differences between Macroeconomics and Microeconomics. How are these two fields related? 

\textbf{Answer:} 

Microeconomics is the study of individual choices and the study of group behaviour in individual markets.

Macroeconomics is the study of broader aggregations of markets / the economy as a whole.

As aggregated variables are the sum of variables describing many individual decisions, macroeconomic analyses often rest on a microeconomic foundation. \\

\textbf{Question 7:} What are the main goals of economic policy? Which decision makers carry out economic policy? 

\textbf{Answer:} 
\begin{itemize}

\item Economic growth

\item Price Stability

\item High Employment

\item Avoidance of a continuous current account deficit or surplus(current account = exports – imports) \\

\end{itemize}

\textbf{Question 8:} In economic models we usually have two sorts of variables. What type of variables are meant and what are the differences between these variables? 

\textbf{Answer:} 

Endogenous and exogenous variables.

Endogenous variables are explained by the model. Exogenous models are those a model takes as given.

The purpose of a model is to show how the exogenous variables affect the endogenous variables. \\

\textbf{Question 9:} Why do economists build models? 

\textbf{Answer:} 

The model illustrates the essence of the real objective it is designed to resemble.

Economists also use models to understand the world.

They, for example, built models which are supposed to explain the relationship between output growth, inflation and unemployment. An economist’s model often uses mathematical symbols and equations. They help us to distinguish irrelevant details from important connections. \\

\textbf{Question 10:} What is a market clearing model? When is it appropriate to assume that markets clear? 

\textbf{Answer:} 

Market clearing model is a model in which quantity supplied = quantity demanded. It is appropriate to assume markets clear when we want to study the change in prices given a change from other factors. \\

\textbf{Question 11:} Use the model of supply and demand to explain how a fall in the price of lemonade would affect the price of cola and the quantity of cola sold. In your explanation, identify the exogenous and endogenous variables. 

\textbf{Answer:} 

Lemonade and Cola are substitutes. If the price of lemonade falls, people will tend to switch to lemonade. Therefore, the demand curve for Cola will shift to the left. The result is that both price and quantity demanded for Cola go down.

In this model, the price of lemonade is exogenous variable, which is taken as given. The price and quantity demand of Cola are endogeneous variables. \\

\textbf{Question 12:} What do you think how often does the price of a haircut change? What does your answer imply about the usefulness of market--clearing models for analysing the market for haircuts? 

\textbf{Answer:} 

Price of haircut does not change often, i.e. the price is sticky in the short run. Given the market clearing assumption, we will see later that there are demand/supply surplus/deficit in the short run. \\

\centering
END

\end{document}
