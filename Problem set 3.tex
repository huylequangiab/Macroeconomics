\documentclass[a4paper, 11pt]{article}

%% Language and font encodings
\usepackage[english]{babel}
\usepackage[utf8x]{inputenc}
\usepackage[T1]{fontenc}
\usepackage{libertine} 

%% Sets page size and margins
\usepackage[a4paper,top=2cm,bottom=2cm,left=3cm,right=3cm,marginparwidth=1.75cm]{geometry}

%% Useful packages
\usepackage{amsmath}
\usepackage{graphicx}
\usepackage[colorinlistoftodos]{todonotes}
\usepackage[colorlinks=true, allcolors=blue]{hyperref}
\usepackage{booktabs} 
\usepackage{multirow}

\title{Problem Set 3}
\author{Huy Le Quang}

\begin{document}
\maketitle

\section {A long run perspective on the economy}

\textbf{Question 1:} What determines the amount of output an economy produces?

\textbf{Answer:} 

The amount of output an economy can produce depends on the two most important input factors: Labor and Capital. In addition, another determinant is the production technology that determines how much firms can produce given amounts of capital (K) and labor (L).

Production function: $Y = F(K,L)$ \\

\textbf{Question 2:} Explain how a competitive, profit-maximizing firm decides how much of each factor of production to demand.

\textbf{Answer:} 

Profit = Revenue - Costs

$\pi = P * F(K,L) - wL - iK$

FOC:

$P * \frac{dF}{dK} = P * MPK = i \rightarrow MPK = r$
$P * \frac{dF}{dL} = P * MPL = w \rightarrow MPL = \frac{w}{P}$

The firms will demand the quantity of labor which equates the Marginal Product of Labor (MPL) and real wage, similarly, they will demand the quantity of capital which equates the Marginal Product of Capital (MPK) and rent. \\

\textbf{Question 3:} Consider an economy with the following Cobb–Douglas production function:
$Y = AK^{1/3}L^{2/3}$; $A=1$.
Derive the equations describing capital demand (= demand for loanable funds = investment function) in this economy as a function of the real interest rate and labour.

\textbf{Answer:}

Profit maximizing firm considers:
$\pi = P * F(K,L) - wL - iK$

FOC:

$\frac{1}{3}K^{-2/3}L^{2/3} = r$
$\rightarrow K^{-2/3} = 3rL^{-2/3}$
$\rightarrow K = 3rL$ \\


\textbf{Question 4:} Using the above version of the Cobb–Douglas production function, calculate the capital income and the labour income in the economy. Interpret your results.

\textbf{Answer:}

Capital income = MPK * K = $\frac{1}{3}K^{1/3}L^{2/3} = \frac{1}{3}Y$

Labor Income = MPL * L = $\frac{2}{3}K^{1/3}L^{2/3} = \frac{2}{3}Y$

Capital accounts for 1/3 income and Labor accounts for 2/3 income. \\


\textbf{Question 5:} What determines consumption and investment?

\textbf{Answer:}
Real interest rate \\

\textbf{Question 6:}: Explain the difference between government purchases and transfer payments. Give two examples of each.

\textbf{Answer:} 

Government purchases are a measure of the dollar value of goods and services purchased directly by the government. For example, the government buys missiles and tanks, builds roads, and provides services such as air traffic control. All of these activities are part of GDP. Transfer payments are government payments to individuals that are not in exchange for goods or services. They are the opposite of taxes: taxes reduce household disposable income, whereas transfer payments increase it. Examples of transfer payments include Social Security payments to the elderly, unemployment insurance, and veterans' benefits.
\\

\textbf{Question 7:} What makes the demand for the economy’s output of goods and services equal the supply?

\textbf{Answer:} 
The real interest rate is the only endogenous variable in the model it must adjust to ensure that the demand for goods equals the supply. \\

\textbf{Question 8:} Explain what happens to consumption, investment, and the interest rate when the government increases taxes.

\textbf{Answer:} 

When the Government increases tax, and Y = const, the disposable income $\downarrow$

$\rightarrow$ Consumption decrease: $\delta C(\overline{Y} - \overline{T}) = \delta T * (-MPC)$

$\rightarrow$ National saving increase: S = Y - C - G $\rightarrow$ Supply for loanable fund shifts to the right

$\rightarrow$ Interest rate $\downarrow$ and Investment $\uparrow$
\\

\textbf{Question 9:} The government raises taxes by \$100 billion. If the marginal propensity to consume is 0.6, what happens to the following? Do they rise or fall? By what amounts?
a) Public saving.
b) Private saving.
c) National saving.
d) Investment.

\textbf{Answer:} 
a. Public saving increases by \$100 billion

b. Private saving increase by 100bil * 0.6 = 60bil

c. National saving increase by 160bil

d. Investment: increase, however, we cannot say the exact amount because it depends on the elasticity of the demand curve for loanable fund. \\


\textbf{Question 10:} Suppose that an increase in consumer confidence raises consumers’ expectations of future income and thus the amount they want to consume today. This might be interpreted as an upward shift in the consumption function. How does this shift affect investment and the interest rate?

\textbf{Answer:} 

Consumption increase $\rightarrow$ Saving decreases $\rightarrow$ Supply for loanable fund shifts to the left $\rightarrow$ interest rate increase and Investment decrease
 \\

\textbf{Question 11:} Consider an economy described by the following equations:
$$Y=C+I+G$$
$$Y=5,000$$
$$G=1,000$$
$$T=1,000$$
$$C=250 +0.75(Y −T)$$
$$I=1,000 −50r$$

\textbf{Answer:} 

a) In this economy, compute private saving, public saving, and national saving. \\
b) Find the equilibrium interest rate.
c) Now suppose that G rises to 1,250. Compute private saving, public saving, and national saving.
d) Find the new equilibrium interest rate.

\textbf{Question 12:} Suppose that the government increases taxes and government purchases by equal amounts. What happens to the interest rate and investment in response to this balanced-budget change? Does your answer depend on the marginal propensity to consume?

\textbf{Answer:} 


\textbf{Question 13:} When the government subsidizes investment, such as with an investment tax credit, the subsidy often applies to only some types of investment. This question asks you to consider the effect of such a change. Suppose there are two types of investment in the economy: business investment and residential investment. And suppose that the government institutes an investment tax credit only for business investment.

\textbf{Answer:}

a) How does this policy affect the demand curve for business investment and the demand curve for residential investment?
b) Draw the economy’s supply and demand for loanable funds. How does this policy af-fect the supply and demand for loanable funds? What happens to the equilibrium in-terest rate?
c) Compare the old and the new equilibrium.
d) How does this policy affect the total quantity of investment? The quantity of business investment? The quantity of residential investment?


\textbf{Question 14:} If consumption depended on the interest rate, how would that affect the conclusions reached in this chapter about the effects of fiscal policy?

\textbf{Answer:}  

\section{Saving and Investment in a Small Open Economy}

\textbf{Question 15:} If a small open economy cuts defence spending, what happens to saving, investment, the trade balance, and the interest rate?

\textbf{Answer:}  

\textbf{Question 16:} If a small open economy bans the import of Japanese DVD players, what happens to saving, investment, the trade balance, and the interest rate?

\textbf{Answer:} 


\textbf{Question 17:} What would happen to the trade balance of a small open economy when government purchases increase, such as during a war? Does you answer depend on whether it is a local war or a world war?

\textbf{Answer:} 


\textbf{Question 18:} Suppose that some foreign countries begin to subsidize investment by instituting investment tax relief.

\textbf{Answer:} 

a) What happens to world investment demand as a function of the world interest rate?
b) What happens to the world interest rate?
c) What happens to investment in our small open economy?
d) What happens to our trade balance?

\section{The Determinants of the Real Exchange Rate}

\textbf{Question 19:} Use the model of the small open economy to predict what would happen to the trade balance, the real exchange rate, and the nominal exchange rate in response to each of the following events.

\textbf{Answer:} 
a) A fall in consumer confidence about the future induces consumers to spend less and save more.
b) A tax reform increases the incentive for businesses to build new factories.
c) The introduction of a stylish line of Toyotas makes some consumers prefer foreign cars over domestic cars.
d) The central bank doubles the money supply.
e) New regulations restricting the use of credit cards increase the demand for money.

\textbf{Question 20:} Consider an economy described by the following equations:
$$Y = C + I + G + NX$$
$$Y = 5,000$$
$$G = 1,000$$
$$T = 1,000$$
$$C = 250 + 0.75(Y − T)$$
$$I = 1,000 − 50r$$
$$NX = 500 – 500\epsilon$$
$$r = r^{∗} = 5$$

\textbf{Answer:} 

a) In this economy, solve for national saving, investment, the trade balance, and the equilibrium exchange rate.
b) Suppose now that G rises to 1,250. Solve for national saving, investment, the trade balance, and the equilibrium exchange rate. Explain what you find.
c) Now suppose that the world interest rate rises from 5 to 10 percent. (G is again 1,000.) Solve for national saving, investment, the trade balance, and the equilibrium exchange rate. Explain what you find.

\textbf{Question 21:} What happens to the real exchange rate in exercises 15-18?

\textbf{Answer:} 


\textbf{Question 22:} In 2005, Federal Reserve Governor Ben Bernanke said in a speech: “Over the past decade a combination of diverse forces has created a significant increase in the global supply of saving—a global saving glut—which helps to explain both the increase in the U.S. current ac-count deficit [a broad measure of the trade deficit] and the relatively low level of long-term real interest rates in the world today.” Is this statement consistent with the models you have learned? Explain.

\textbf{Answer:} 

\section{The Supply and Demand for Money}

\textbf{Question 23:} What does the assumption of constant velocity imply?

\textbf{Answer:} 


\textbf{Question 24:} What is an inflation tax and who pays it?

\textbf{Answer:} 


\textbf{Question 25:} Name two channels which determine the demand for money. Explain.

\textbf{Answer:} 



\centering
END

\end{document}
