\documentclass[a4paper, 11pt]{article}

%% Language and font encodings
\usepackage[english]{babel}
\usepackage[utf8x]{inputenc}
\usepackage[T1]{fontenc}
\usepackage{libertine} 

%% Sets page size and margins
\usepackage[a4paper,top=2cm,bottom=2cm,left=3cm,right=3cm,marginparwidth=1.75cm]{geometry}

%% Useful packages
\usepackage{amsmath}
\usepackage{graphicx}
\usepackage[colorinlistoftodos]{todonotes}
\usepackage[colorlinks=true, allcolors=blue]{hyperref}
\usepackage{booktabs} 
\usepackage{multirow}

\title{Problem Set 2}
\author{Huy Le Quang}

\begin{document}
\maketitle

\section {Gross Domestic Product - GDP}

\textbf{Question 1:} Define the GDP verbally. What are the differences between the real and the nominal GDP?

\textbf{Answer:} 

Gross Domestic Product - GDP: measures the market value $P_{it}Y_{it}$ of the sum of all final goods and services produced within an economy in a given period of time $t$

Nominal GDP measures the value of the output of the economy at current prices

Real GDP measures the value of the output of the economy using constant prices (i.e the prices of the selected base year) \\

\textbf{Question 2:} Explain why the nominal GDP is not (always) a proper indicator to describe the aggregate economic development. 

\textbf{Answer:} 

The change in GDP over time could be both due to changes in price level and changes in quantity of products. If we want to know whether economic well--being has increased overtime, nominal GDP using current prices is not an appropriate measure of this issue. \\

\textbf{Question 3:} How is the growth rate of the GDP defined?

\textbf{Answer:}

Growth rate is the increase in the market value of goods and services produced by an economy over time (e.g. from year to year).

 % Table generated by Excel2LaTeX from sheet 'Tabelle1'
\begin{table}[htbp]
  \centering
  \caption{Growth rate in 2009}
    \begin{tabular}{lrr}
    \toprule
    \toprule
    \multicolumn{1}{c}{\textbf{Year}} & \multicolumn{1}{c}{\textbf{2008}} & \multicolumn{1}{c}{\textbf{2009}} \\
    \midrule
    \textbf{Real GDP} &                         2,477.8    &        2,435.9    \\
    \textbf{Growth rate} &       & -1.69\% \\
    \textbf{Growth rate (approx)} &       & -1.71\% \\
    \bottomrule
    \end{tabular}%

\end{table}% 



\textbf{Question 4:} Calculation

% Table generated by Excel2LaTeX from sheet 'Tabelle1'
\begin{table}[htbp]
  \centering

    \begin{tabular}{lrr}
    \toprule
    \toprule
          & \multicolumn{1}{c}{\textbf{Apples}} & \multicolumn{1}{c}{\textbf{Orange}} \\
    \midrule
    Q1    & 50    & 100 \\
    P1    & 1     & 0.8 \\
    Q2    & 80    & 120 \\
    P2    & 1.25  & 1.6 \\
    \bottomrule
    \end{tabular}%
  \label{tab:addlabel}%
\end{table}%


\textbf{Answer:}
% Table generated by Excel2LaTeX from sheet 'Tabelle1'
\begin{table}[htbp]
  \centering
  \caption{Nominal, real GDP and growth rate}
    \begin{tabular}{lrr}
    \toprule
    \toprule
          & \multicolumn{1}{c}{\textbf{Year 1}} & \multicolumn{1}{c}{\textbf{Year 2}} \\
    \midrule
    Nominal GDP & 130   & 292 \\
    Real GDP (Year 1 as base) & 130   & 176 \\
    Real GDP (Year 2 as base) & 222.5 & 292 \\
    Growth rate (Year 1 as base) &       & 35.4\% \\
    Growth rate (Year 2 as base) &       & 31.2\% \\
    Geometric mean (Year 1 as base) &       & 0.16 \\
    Geometric mean (Year 2 as base) &       & 0.15 \\
    \bottomrule
    \end{tabular}%

\end{table}%

\section {Macroeconomic Accounting}

\textbf{Question 5:} The GDP can be defined from three different points of view. List the three things the GDP measures.

\textbf{Answer:}

\begin{itemize}
\item Output/Product approach
\item Total Expenditure approach
\item Total Income approach
\end{itemize}  

\textbf{Question 6:}: What is the difference between the Gross Domestic Product (GDP) and the Gross National Product (GNP)? What is the difference between the Net Exports (NX) and the Current Account (CA)? What’s going on if $NX>0$ and $CA=0$?

\textbf{Answer:} 

GNP = GDP + Factor Payments from Abroad - Factor Payments to Abroad
Current Account = NX + Net Factor Payments from Abroad (NFIA) + unilateral transfer
If $NX>0$ and $CA=0$, then NFIA must be smaller than 0. This means that we export more to abroad than import products from abroad, and at the same time, we receive more payments from abroad to buy our exported products. \\

\textbf{Question 7:} What happens to GDP when the owner of a small firm marries her secretary and stops paying for his work, which he continuous to perform?

\textbf{Answer:} 
Nothing changes in GDP because either he pays himself as his salary or the money is retained in his firm's profit. \\

\textbf{Question 8:} I bought my house for €80,000. I have just sold it for €110,000, and the estate agent received a 10\% commission from the buyer. What is the effect on GDP?

\textbf{Answer:} 

Increase in GDP $= (110,000-80,000) + 110,000*0.1 = 30,000 + 11,000 = 41,000$ \\

\textbf{Question 9:} Define the term value added.

\textbf{Answer:} 

Value added is the difference between the value of the final output sold and the cost of purchasing the raw materials and intermediate goods needed to produce the output. \\

\textbf{Question 10:} A farmer grows a kilo of wheat and sells it to a miller for € 1. The miller turns the wheat into flour and then sells the flour to a baker for €3. The baker uses it to make bread and sells the bread for € 6 to an engineer who eats the bread. What is the value added by each person? What is GDP?

\textbf{Answer:} 

Value added if the farmer = €1
Value added of the miller = €2
Value added of the baker = €3
GDP = €6  \\

\textbf{Question 11:} What are the components of aggregate demand?

\textbf{Answer:} 

$$AD = C + I + G + X$$ \\

\textbf{Question 12:} Place each of the following transactions in one of the four components of expenditure for the German economy: consumption, investment, government purchases, and net exports: (a) BMW sells a car to a German household; (b) BMW sells a car to a US resident; (c) BMW sells a car to the German government; (d) BMW produces a car to be sold next year.

\textbf{Answer:} 
\begin{itemize}
\item a. BMW sells a car to a German household: Consumption
\item b. BMW sells a car to a US resident: Net export
\item c. BMW sells a car to the German government: Government purchases
\item d. BMW produces a car to be sold next year: Investment   \\
\end{itemize}

\textbf{Question 13:} How would the following transactions affect UK net capital outflow?

\textbf{Answer:}

Net capital outflow =  $S^{N} - I$
Net capital outflow = Amount domestic residents are lending to foreigners - Amount foreigners are lending to domestic residents = Outflow - Inflow

\begin{itemize}

\item a. A British mobile telephone company establishes an office in the Czech Republic: Outflow increases, Net capital outflow increases
\item b. An US company’s pension fund buys shares in BP: Inflow increases, Net capital outflow decreases
\item c. Toyota expands its factory in Derby, England: Inflow increases, Net capital outflow decreases
\item d. A London-based investment trust sells its Volkswagen shares to a French investor: Inflow increases, Net capital outflow decreases. 
\end{itemize}

\textbf{Question 14:} Holding national saving constant: Does an increase in net capital outflow increase, decrease or have no effect in a country's accumulation of domestic capital (stock)?

\textbf{Answer:}  

Net capital outflow = $S^{N} - I$
$\rightarrow I =  S^{N}$ - Net capital outflow
If $S^{N} = const$, 
\begin{itemize}
\item Net capital outflow $\uparrow \rightarrow I \downarrow$
\item Net capital outflow $\downarrow \rightarrow I \uparrow$
\end{itemize}

\textbf{Question 15:} Consider the following identity: $S^{P} – I = NX – S^{G}$. Show that the identity holds, explain what the identity means.

\textbf{Answer:}  
We have Income = Expenditure
$$C + S^{P} + T = C + I + G + (X - M)$$
$$\rightarrow S^{P} - I = -(T - G) + (X - M)$$
$$\rightarrow S^{P} - I = NX - S^{G}$$

$\rightarrow$ Private net saving = Trade balance - Government saving

\section{Consumer Price Indices, Inflation and Deflation}

\textbf{Question 16:} What does the consumer price index measure? How is it defined, and what is the differences between the CPI and the GDP-Deflator?

\textbf{Answer:} 
Consumer Price Index (CPI) measures the current price of a base year basket of goods and services relative to the price of the same basket in the base year

$$CPI_{t} = \frac{\sum_{j}P_{jt}Y_{j0}}{\sum_{j}P_{j0}Y_{j0}}$$

Difference between CPI and GDP Deflator:
\begin{itemize}
\item Deflator measures value of all goods, CPI considers only the goods in the consumer basket.
\item Deflator measures only goods that are domestically produced, imported goods are not considered
\item The CPI uses fixed basket whereas the deflator allows the basket to vary.
\end{itemize}

\textbf{Question 17:} Define the term inflation rate. What is the differences between inflation, deflation and disinflation?

\textbf{Answer:} 
Inflation occurs when there is a significant and sustained increase in price index ($\pi>>0$)

Deflation occurs when there is a sustained decrease in consumer prices ($\pi<0$)

Disinflation is the short term slow down in the rate of inflation

\textbf{Question 18:} If the CPI in 2009 is 107 and 108.4 in 2010. Calculate the rate of inflation in 2010.

\textbf{Answer:} 

$$ \pi_{2010} = \frac{(108.4 - 107)}{107} = 1.31\%$$

\textbf{Question 19:} Calculation 

\textbf{Answer:} 
% Table generated by Excel2LaTeX from sheet 'Tabelle1'
\begin{table}[htbp]
  \centering
  \caption{Answer to questions in a and b}
    \begin{tabular}{lrr}
    \toprule
    \toprule
          & \multicolumn{1}{c}{\textbf{2010}} & \multicolumn{1}{c}{\textbf{2015}} \\
    \textbf{Prices} &       &  \\
    \multicolumn{1}{r}{Car} &                           50,000    &            60,000    \\
    \multicolumn{1}{r}{Bread} &                                  10    &                    20    \\
    \textbf{Quantity} &       &  \\
    \multicolumn{1}{r}{Car} &                                100    &                  120    \\
    \multicolumn{1}{r}{Bread} &                        500,000    &          400,000    \\
    \midrule
    Nominal GDP &                   10,000,000    &    15,200,000    \\
    Real GDP &                   10,000,000    &    10,000,000    \\
    GDP deflator &                                100    &                  152    \\
    CPI   &                               1.00    &                1.60    \\
    Price increase (Laspeyres) &       & 60\% \\
    Price increase (Paasche) &       & 52\% \\
    \bottomrule
    \end{tabular}%
 
c. If the Government wants to index pensions, we should use GDP deflator because pensions is not an item in the consumer basket.
\end{table}%


\textbf{Question 20:} Calculation

\textbf{Answer:} 

% Table generated by Excel2LaTeX from sheet 'Tabelle1'
\begin{table}[htbp]
  \centering
  \caption{Answers to questions from a to d}
    \begin{tabular}{lrr}
    \toprule
    \toprule
          & \multicolumn{1}{c}{\textbf{Year 1}} & \multicolumn{1}{c}{\textbf{Year 2}} \\
    \textbf{Prices} &       &  \\
    \multicolumn{1}{r}{Red} & 1     & 2 \\
    \multicolumn{1}{r}{Green} & 2     & 1 \\
    \textbf{Basket} &       &  \\
    \multicolumn{1}{r}{Red} & 10    & 0 \\
    \multicolumn{1}{r}{Green} & 0     & 10 \\
    \midrule
    CPI (Base Year 1) & 1     & 2 \\
    Nominal Spending & 10    & 10 \\
    Real Spending (Base Year 1) & 10    & 20 \\
    Price deflator & 100   & 50 \\
    \bottomrule
    \end{tabular}%
\end{table}%

\clearpage

e. Suppose H is equally happy eating red or green apples. The true cost of living does not change for H because he switches between red and green apple when their prices change.

\textbf{Question 21:} What is the difference between the real and nominal interest rate according to the Fisher equations?

\textbf{Answer:} 
\begin{itemize}
\item Nominal interest rate is the interest rate that banks pay
\item Real interest rate is the increase in your purchasing power
\item Fisher equation: $i = r + \pi$
\end{itemize}

\textbf{Question 22:} Define the terms ex-ante and ex-post and give an example.

\textbf{Answer:} 

\begin{itemize}
\item Ex-ante is something happens before (EX: expected inflation)
\item Ex-post is something happens after (when something is realized) (EX: realized inflation)
\end{itemize}

\textbf{Question 23:} List different costs of inflation.

\textbf{Answer:} 
\begin{itemize}
\item Shoe leather cost
\item Menu cost
\item Inflation induced tax distortions
\end{itemize}

\textbf{Question 24:} Define the term real and nominal variable and give an example of each.

\textbf{Answer:} 


\textbf{Question 25:} Statement: Inflation is reallocation of debt. What is meant by this? Does it matter whether the inflation is expected or unexpected? Explain!

\textbf{Answer:} 

Example: debtor and creditor enter into a long-term loan agreement. They both expect that inflation is low in the future, so they agree to set a low interest rate on this loan. Suppose that inflation is unexpectedly increasing over the next years, (1) the creditor receives a low interest payment because he already signed the contract and cannot increase the interest rate; (2) the real purchasing power of money is decreasing, which makes it even worse for creditor. In this case, creditor will lose, and debtor will gain. We can imagine a reverse situation if inflation decreases in the future. Therefore, unexpected inflation is a problem because it reallocates the wealth of debtor and creditor. \\

\section{Money and Monetary Aggregates}

\textbf{Question 26:} What is the difference between fiat and commodity money?

\textbf{Answer:} 

A fiat money is a legal claim as it attains all its properties from the law. 

Commodity money, on the other hand, is money that derives its value from a commodity of which it is made. \\

\textbf{Question 27:} Describe the functions of money in an economy. Which of these functions do the following items (not) satisfy? (a) a credit card; (b) a painting; (c) a bus ticket.

\textbf{Answer:} 

Functions of money in the economy:
\begin{itemize}
\item a stock of assets that store values: credit card, bus ticket, painting
\item a unit of account
\item a medium of exchange: painting
\end{itemize}

\textbf{Question 28:} What are open-market operations, and how do they influence the money supply?

\textbf{Answer:} 

OMOs are the purchase and sale of non-monetary assets (mainly government bonds)from and to the commercial banks by the central bank to expand or contract the money in the banking system.

To expand the money supply: central bank buys assets and pay them with ``newly'' printed money.

To reduce the money supply: central bank sells assets and receive money (and ``destroy'' it). \\

\textbf{Question 29:} What is the money market? What is the refinancing rate and how is it related to the money market?

\textbf{Answer:} 

Money market is where financial instruments with high liquidity and very short maturities are traded.

Refinancing rate is the rate at which commercial banks borrow from the central bank. Normally, these borrowings are extremely short term (overnight), therefore, the transactions happen in the money market.

\textbf{Question 30:} Write down the quantity equation of money and explain it.

\textbf{Answer:} 

Quantity equation of money: $M x v = P x Y$

This equation describe how the amount of money M in an economy is related to the number of economic transactions (= the economy's nominal output $PY$). $v$ is the velocity of money (number of times an euro bill changes hands in a given time period). \\

\textbf{Question 31:} Explain how (private) banks create money.

\textbf{Answer:} 

In a fractional reserve system, banks create money. Because commercial banks only keep a fraction of their deposits in reserve, and lend out the rest. The next bank receiving this amount of money will continue to keep only a fraction of deposits in reserve and lend out the rest. This process continues, and the total money supply created depends on the reserve ratio: Total money supply = $\frac{1}{rr}$ x Original Deposits. \\

\textbf{Question 32:} What are the various ways in which the central bank can influence the money supply?

\textbf{Answer:} 
\begin{itemize}
\item Open market operations
\item Change reserve requirements
\item Change refinancing rate
\end{itemize}

\textbf{Question 33:} Explain how each of the following events changes the money supply (if at all).

\textbf{Answer:} 

\begin{itemize}
\item a. The central bank buys bonds in an open market operation: Increase money supply
\item b. The central bank increases the interest rate it pays banks holding reserves: Decrease money supply
\item c. Rumours about a computer virus attack on ATMs increase the amount of money people hold as currency rather than demand deposits: Decrease money supply
\item d. The ECB flies a helicopter over the Frankfurt pedestrian precinct and drops 100 euro bills: Increase money supply
\end{itemize}

\textbf{Question 34:} Why might a banking crisis lead to a fall in money supply?

\textbf{Answer:} 
The change in money supply is determined by the (deposit) behavior of households and (lending) behavior of commercial banks. When financial crisis happens, households lose trust in the banking system so they want to hold money (at home) rather than depositing in the banks $\rightarrow$ Money supply falls. Business do not have profitable projects, so they do not need to borrow from banks. If banks cannot lend out money, they cannot create money $\rightarrow$ Money supply falls.

\section{Unemployment}
\textbf{Question 35:} Assume that in the UK in 2010 of all adult people 28,980,000 were employed, 1,116,000 were unemployed and around 10,904,000 were not in the labour force. How big was the labour force? What was the labour force participation rate? What was the unemployment rate?

\textbf{Answer:}
Labor force = $28,980,000 + 1,116,000 = 30,096,000$

Labor force participation rate = $\frac{30,096,000}{30,096,000 + 10,904,000} = 73.4\%$

Unemployment rate = $\frac{1,116,000}{30,096,000} = 3.71\%$ 

\textbf{Question 36:} Name four different reasons why unemployment may occur and briefly describe the differences between the different kinds of unemployment.

\textbf{Answer:}

\textbf{Question 37:} Give three explanations for wages rigidities – why the real wage rate may remain above the level that equilibrates labour supply and labour demand.

\textbf{Answer:}

\textbf{Question 38:} What causes cyclical unemployment? 

\textbf{Answer:}

\textbf{Question 39:} What is the effect on wages, employment, and unemployment of a wave of immigra-tion: (a) in the absence of minimum wage legislation? (b) in the presence of (a binding) minimum wage legislation?

\textbf{Answer:}

\textbf{Question 40:} Suppose that a country experiences a reduction in productivity.

\textbf{Answer:}

\begin{itemize}
\item a. What happens to the labour demand curve?
\item b. What is the effect on wages, employment, and unemployment – if the labour market were always in equilibrium?
\item c. How would this change in productivity affect the labour market if unions pre-vented real wages from falling?
\end{itemize}

\textbf{Question 41:} Consider an economy with the following Cobb-Douglas production function: $Y = K^{\frac{1}{3}}L^{\frac{2}{3}}$. The economy has 1,000 units of capital and a labour force of 1,000 workers.

\textbf{Answer:}
a. Derive the equation describing labour demand in this economy as a function of the real wage and the capital stock.
b. Assume that the 1,000 units of capital and labour are supplied in-elastically. If the real wage is flexible and adjusts to equilibrate supply and demand on the labour market, what is the real wage? In that equilibrium, what are the employment, output and the total amount owned by workers?
c. Now suppose the government passes a law to pay workers a real wage of 1 unit of output. How does this wage compare to the equilibrium wage?
d. What do you think are the effects of this wage?

\textbf{Question 42:} Find the optimal supply of work (= leisure demand) for a real wage of € 20/hr for someone who views consumption and leisure as perfect complements in a 10 to 1 ratio (i.e. she requires 1 hr of leisure for every € 10 of consumption). The maximum amount of leisure is 24 hr.

\textbf{Answer:}

\textbf{Question 43:} Find the optimal supply of work (= leisure demand) for a real wage of € 20/hr for someone who views consumption and leisure as perfect substitutes in a 10 to 1 ratio (i.e. she is willing to sacrifice 1 hr of leisure for € 10 of consumption). The maximum amount of leisure is 24 hr.

\textbf{Answer:}

\textbf{Question 44:} Individual labour supply may be a downward-sloping curve for many individuals, for others it maybe downward sloping but backward-bending at some real-wage level. Why can we still confidentially presume that aggregate labour supply is upward sloping?

\textbf{Answer:}

\textbf{Question 45:} Most countries levy taxes on labour. What is the effect of these taxes on labour supply and equilibrium wages? Assume that workers care for after tax wages only. State your assumptions carefully.

\textbf{Answer:}

\textbf{Question 46:} It is well known in Europe that the unemployment rate among well-educated is con-siderably lower than for those who left school early or without training. Why might this be the case? How might an “education offensive” help solve the unemployment problem, even if wages for the low-skilled are rigid?

\textbf{Answer:}

\textbf{Question 47:} Are the following workers more likely to experience short-term or long-term unem-ployment? Explain.

\textbf{Answer:}
\begin{itemize}
\item a. A construction worker laid off because of bad weather.
\item b. A manufacturing worker who loses her job at a plant in an isolated area.
\item c. A bus industry worker laid off because of competition from the railway.
\item d. An expert welder with little formal education who loses her job when the company installs automatic welding machinery.

\end{itemize}

\textbf{Question 48:} The residents of a certain hall of residence have collected the following data: people who live in the hall can be classified as either involved or uninvolved people, 10 per cent experience a break-up of their relationship every month. Among uninvolved people, 5 percent will enter into a relationship every month. What is the steady-state fraction of residents who are uninvolved?

\textbf{Answer:}

\textbf{Question 49:} Suppose the government passes legislation making it more difficult for firms to fire workers. If the legislation reduces the rate of job separation without affecting the rate of job finding, how would the natural rate of unemployment change?

\textbf{Answer:}

\textbf{Question 50:} In any city at any time, some of the stock of usable office space is vacant. This vacant office space is unemployed capital. How would you explain this phenomenon? Is it a social problem?

\textbf{Answer:}

\section{Exchange rate}

\textbf{Question 51:} Define the terms nominal and real exchange rate. What is the difference between the two expressions?

\textbf{Answer:}

\textbf{Question 52:} What happens to the real exchange rate between two countries if the price level at home doubles, all other things equal (= ceteris paribus)? What if – ceteris paribus – the price of foreign goods doubles? What if – ceteris paribus – the nominal exchange rate doubles?

\textbf{Answer:}

\textbf{Question 53:} What is happening to the Swiss real exchange rate in each of the following situations? Explain.

\textbf{Answer:}
\begin{itemize}
\item a. The Swiss nominal exchange rate is unchanged, but prices rise faster in Switzer-land than abroad.
\item b. The Swiss nominal exchange rate is unchanged, but prices rise faster abroad than in Switzerland.
\item c. The Swiss nominal exchange rate declines, and prices are unchanged in Switzer-land and abroad.
\item d. The Swiss nominal exchange rate declines, and prices rise faster abroad than in Switzerland:
\end{itemize}


\textbf{Question 54:} Suppose a friend tells you that travelling in Mexico is much cheaper now than it was five years ago. ‘Five years ago,’ says your friend, ‘a UK pound bought 10 pesos; this year, a pound buys 15 pesos.’ Is your friend right or wrong? Given that total inflation over this period was about 25 per cent in the United Kingdom and 100 per cent in Mexico, has it become more or less expensive for UK residents to travel in Mexico? Write your answer using a concrete example - such as the price of a British pork pie versus the price of a Mexican taco - that will convince your friend.

\textbf{Answer:}

\textbf{Question 55:} List two goods for which the law of one price is likely to hold, and two goods for which it probably does not hold. Justify your choices.

\textbf{Answer:}

\textbf{Question 56:} A can of lemonade is priced at € 0.75 in Europe and 12 pesos in Mexico. What would the peso-euro exchange rate be if absolute purchasing power parity holds? If a mone-tary expansion caused all prices in Mexico to double, so that lemonade rose to 24 pe-sos, what would happen to the peso-euro exchange rate (if PPP holds)?

\textbf{Answer:}

\textbf{Question 57:} Assume that American rice sells for $1 a kilo, Japanese rice sells for 160 yen a kilo, and the nominal exchange rate is 80 yen per dollar.

\textbf{Answer:}

a. Explain how you could make a profit from this situation. What would be your profit per kilo of rice? If other people exploit the same opportunity, what would happen to the price of rice in Japan and the price of rice in the US?
b. Suppose that rise is the only commodity in the world. What would happen to the real exchange rate between the United States and Japan?

\textbf{Question 58:} How did Bela Balassa and (1928-1991) and Paul Samuelson (1915-2009, Nobel laureate from MIT) explain why wealthier countries are systematically more expensive than poorer ones?

\textbf{Answer:}


\centering
END

\end{document}
