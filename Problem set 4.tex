\documentclass[a4paper, 11pt]{article}

%% Language and font encodings
\usepackage[english]{babel}
\usepackage[utf8x]{inputenc}
\usepackage[T1]{fontenc}
\usepackage{libertine} 

%% Sets page size and margins
\usepackage[a4paper,top=2cm,bottom=2cm,left=3cm,right=3cm,marginparwidth=1.75cm]{geometry}

%% Useful packages
\usepackage{amsmath}
\usepackage{graphicx}
\usepackage[colorinlistoftodos]{todonotes}
\usepackage[colorlinks=true, allcolors=blue]{hyperref}
\usepackage{booktabs} 
\usepackage{multirow}

\title{Problem Set 4}
\author{Huy Le Quang}

\begin{document}
\maketitle

\section {The Model of Aggregate Supply and Aggregate Demand}

\textbf{Question 1:} Give an example of a price that is sticky in the short run and flexible in the long run.

\textbf{Answer:} 

 \\

\textbf{Question 2:} Why does the aggregate demand curve slope downward?

\textbf{Answer:} 

 \\

\textbf{Question 3:} Explain the impact of an increase in the money supply in the short run and in the long run.

\textbf{Answer:}




\textbf{Question 4:} What is it easier for the ECB, to deal with demand shocks or to deal with supply shocks? Explain?

\textbf{Answer:}



\textbf{Question 5:} Suppose that a change in government regulations allows banks to start paying interest on checking accounts. Recall that the money stock M1 is the sum of currency and demand deposits, including checking accounts, so this regulatory change makes holding money more attractive.

\textbf{Answer:}
a) How does this change affect the demand for money?
b) What happens to the velocity of money?
c) If the ECB keeps the money supply constant, what will happen to output and prices in the short run and in the long run?
d) Should the ECB keep the money supply constant in response to this regulatory change? Why or why not?

\textbf{Question 6:}: Suppose the ECB reduces the money supply by 5 percent.

\textbf{Answer:} 
a) What happens to the aggregate demand curve?
b) What happens to the level of output and the price level in the short run and in the long run?
c) According to Okun’s law, what happens to unemployment in the short run and in the long run?
d) What happens to the real interest rate in the short run and in the long run?
\\

\textbf{Question 7:} Let’s examine how the goals of the ECB influence its response to shocks. Suppose ECB A cares only about keeping the price level stable, and ECB B cares only about keeping output and employment at their natural rates. Explain how each ECB would respond to…

\textbf{Answer:} 
a) … an exogenous decrease in the velocity of money?
b) … an exogenous increase in the price of oil? \\

\section{The IS-LM Model}

\textbf{Question 8:} Use the Keynesian cross to explain why fiscal policy has a multiplied effect on national income.

\textbf{Answer:} 

\\

\textbf{Question 9:} Use the theory of liquidity preference to explain why an increase in the money supply lowers the interest rate. What does this explanation assume about the price level?

\textbf{Answer:} 

\\

\textbf{Question 10:} Why does the IS curve slope downward?

\textbf{Answer:} 

  \\

\textbf{Question 11:} Why does the LM curve slope upward?

\textbf{Answer:} 



\textbf{Question 12:}Use the Keynesian cross to predict the impact of
a. An increase in government purchases.
b. An increase in taxes.
c. An equal increase in government purchases and taxes.

\textbf{Answer:} 


\textbf{Question 13:} In the Keynesian cross, assume that the consumption function is given by $C = 200 + 0.75 (Y − T)$. Planned investment is 100; government purchases and taxes are both 100.

\textbf{Answer:}

a. Graph planned expenditure as a function of income.
b. What is the equilibrium level of income?
c. If government purchases increase to 125, what is the new equilibrium income?
d. What level of government purchases is needed to achieve an income of 1,600?


\textbf{Question 14:}Although our development of the Keynesian cross in this chapter assumes that taxes are a fixed amount, in many countries taxes depend on income. Let’s represent the tax system by writing tax revenue as $T = T_{0} + tY$, where $T$ and $t$ are parameters of the tax code. The parameter $t$ is the marginal tax rate: if income rises by €1, taxes rise by $t$ × €1.

\textbf{Answer:}  

a. How does this tax system change the way consumption responds to changes in GDP?
b. In the Keynesian cross, how does this tax system alter the government-purchases multiplier?
c.** In the IS–LM model, how does this tax system alter the slope of the IS curve?

\textbf{Question 15:} Consider the impact of an increase in thriftiness in the Keynesian cross. Suppose the consumption function is $C = C_{0} + c(Y − T)$, where $C_{0}$ is a parameter called autonomous consumption and $c$ is the marginal propensity to consume.

\textbf{Answer:}  
a. What happens to equilibrium income when the society becomes more thrifty, as represented by a decline in $C_{0}$.
b. What happens to equilibrium saving?
c. Why do you suppose this result is called the paradox of thrift?
d. Does this paradox arise in the classical (long-run) model? Why or why not?

\textbf{Question 16:} Suppose that the money demand function is $(M/P)^{d}= 1,000 − 100r$, where $r$ is the in-terest rate in percent. The money supply $M$ is 1,000 and the price level $P$ is 2.

\textbf{Answer:} 
a. Graph the supply and demand for real money balances.
b. What is the equilibrium interest rate?
c. Assume that the price level is fixed. What happens to the equilibrium interest rate if the supply of money is raised from 1,000 to 1,200?
d. If the ECB wishes to raise the interest rate to 7 percent, what money supply should it set?

\textbf{Question 17:} Explain why the aggregate demand curve slopes downward.

\textbf{Answer:} 


\textbf{Question 18:} What is the impact of an increase in taxes on the interest rate, income, consumption, and investment?

\textbf{Answer:} 


\textbf{Question 19:} What is the impact of a decrease in the money supply on the interest rate, income, consumption, and investment?

\textbf{Answer:} 


\textbf{Question 20:} Describe the possible effects of falling prices on equilibrium income.


\textbf{Answer:} 



\textbf{Question 21:} According to the IS–LM model, what happens to the interest rate, income, consumption, and investment under the following circumstances?

\textbf{Answer:} 
a. The central bank increases the money supply.
b. The government increases government purchases.
c. The government increases taxes.
d. The government increases government purchases and taxes by equal amounts.

\textbf{Question 22:}Use the IS–LM model to predict the effects of each of the following shocks on in-come, the interest rate, consumption, and investment. In each case, explain what the ECB should do to keep income at its initial level.

\textbf{Answer:} 

a. After the invention of a new high-speed computer chip, many firms decide to up-grade their computer systems.
b. A wave of credit-card fraud increases the frequency with which people make transactions in cash.)
c. A best-seller titled Retire Rich convinces the public to increase the percentage of their income devoted to saving.

\textbf{Question 23:} Consider the economy of Hicksonia.
a. The consumption function is given by $C = 200 + 0.75(Y − T)$.
The investment function is $I = 200 − 25r$.
Government purchases and taxes are both 100.
For this economy, graph the IS curve for r ranging from 0 to 8.
b. The money demand function in Hicksonia is
$(M/P)^{d} = Y − 100r.
The money supply $M$ is 1,000 and the price level $P$ is 2. For this economy, graph the LM curve for $r$ ranging from 0 to 8.
c. Find the equilibrium interest rate r and the equilibrium level of income Y.
d. Suppose that government purchase are raised from 100 to 150. How much does the IS curve shift? What are the new equilibrium interest rate and level of income?
e. Suppose instead that the money supply is raised from 1,000 to 1,200. How much does the LM curve shift? What are the new equilibrium interest rate and level of income?
f. With the initial values for monetary and fiscal policy, suppose that the price level rises from 2 to 4.What happens? What are the new equilibrium interest rate and level of income?
g. Derive and graph an equation for the aggregate demand curve. What happens to this aggregate demand curve if fiscal or monetary policy changes, as in parts (d) and (e)

\textbf{Answer:} 


\textbf{Question 24:} Explain why each of the following statements is true. Discuss the impact of monetary and fiscal policy in each of these special cases.

\textbf{Answer:} 
a. If investment does not depend on the interest rate, the IS curve is vertical.
b. If money demand does not depend on the interest rate, the LM curve is vertical.
c. If money demand does not depend on income, the LM curve is horizontal.
d. If money demand is extremely sensitive to the interest rate, the LM curve is hori-zontal.

\textbf{Question 25:} Suppose that the government wants to raise investment but keep output constant. In the IS–LM model, what mix of monetary and fiscal policy will achieve this goal? In the early 1980s, the U.S. government cut taxes and ran a budget deficit while the Fed pursued a tight monetary policy. What effect should this policy mix have?

\textbf{Answer:} 

\textbf{Question 26}: The ECB is considering two alternative monetary policies:
$\rightarrow$ holding the money supply constant and letting the interest rate adjust, or
$\rightarrow$ adjusting the money supply to hold the interest rate constant.
In the IS–LM model, which policy will better stabilize output under the following conditions?
a. All shocks to the economy arise from exogenous changes in the demand for goods and services.
b. All shocks to the economy arise from exogenous changes in the demand for money.

\centering
END

\end{document}
